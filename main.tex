%------------------------------------------------------------------------------
% 
% by Quentin Bammey, Marina Gardella, Tina Nikoukhah, Miguel Colom, Jean-Michel Morel, Rafael Grompone von Gioi
%------------------------------------------------------------------------------
\documentclass{ipol}

\ipolSetTitle{Image Forgeries detection through Mosaic Analysis: the Intermediate Values Algorithm}
\ipolSetAuthors{Quentin Bammey, Rafael Grompone von Gioi, Jean-Michel Morel}

\ipolSetAffiliations{Universit\'e Paris-Saclay, ENS Paris-Saclay, CNRS, Centre Borelli, F-94235, Cachan, France \\
\texttt{\{quentin.bammey, rafael.grompone, jean-michel.morel\}@ens-paris-saclay.fr}}

%------------------------------------------------------------------------------
\ipolPreprintLink{https://ipolcore.ipol.im/demo/clientApp/demo.html?id=77777000095}

%------------------------------------------------------------------------------
\usepackage{subcaption}
\usepackage[dvipsnames]{xcolor}

%All the colours we used for the database articles. c0, c1, c2 and c3 go nicely together
\definecolor{c0}{HTML}{0E918C}
\definecolor{c1}{HTML}{F6830F}
\definecolor{c2}{HTML}{BB2205}
\definecolor{c3}{HTML}{1F3C88}
\definecolor{grayA}{HTML}{D2D3C9}
\definecolor{cgray}{HTML}{708090}
\definecolor{cgreen}{HTML}{1F8C88}
\usepackage[vlined,algoruled,linesnumbered]{algorithm2e}
\usepackage{namedinput}

\SetKwNamedIO{Input}{Input}
\SetKwNamedIO{Parameter}{Param}
\SetKwNamedIO{Output}{Output}
%\SetKwInOut{dontusethis}{aaaaaaaaaaaaaa}  % hack to get enough spacing for names
\renewcommand{\KwSty}[1]{\textnormal{\textcolor{c1}{\ttfamily\bfseries #1}}\unskip}
\renewcommand{\ArgSty}[1]{\textnormal{\ttfamily #1}\unskip}
\renewcommand{\DataSty}[1]{\color{c2}\bfseries #1}
\SetKwComment{Comment}{\color{c0}\# }{}
\renewcommand{\CommentSty}[1]{\textnormal{\ttfamily\color{c0}#1}\unskip}
\newcommand{\assign}{:\!=}
\newcommand{\peq}{+\mkern-8mu=}
\newcommand{\var}{\texttt}
\newcommand{\FuncCall}[2]{\texttt{\bfseries #1(#2)}}
\SetKwProg{Function}{function}{}{}
\renewcommand{\ProgSty}[1]{\texttt{\bfseries \color{c3}#1}}
\DontPrintSemicolon
\SetKw{IN}{in}
\SetKw{AND}{and}
\SetKw{FROM}{from}
\SetKw{TO}{to}
%\newcommand{\IN}{~\KwSty{in}~}
%\newcommand{\FROM}{~\KwSty{from}~}
%\newcommand{\TO}{~\KwSty{to}~}
\usepackage{booktabs} % table
\usepackage{multirow}
\usepackage{array}

\usepackage{amsmath, amssymb}
\DeclareMathOperator*{\argmax}{arg\,max}

\renewcommand*{\thesubfigure}{(\thefigure.\arabic{subfigure})}
%------------------------------------------------------------------------------
\newcommand{\qb}[1]{\textcolor{c1}{(Quentin) #1}}

\begin{document}


\begin{ipolAbstract}
Here is the abstract.
\end{ipolAbstract}

\begin{ipolCode}
The reviewed source code and documentation for this algorithm are
available from \href{\ipolLink}{the web page of this
article}. Usage instruction are included in the
\verb|README.txt| file of the archive.
\end{ipolCode}

\begin{ipolSupp}
The link to the demo is \href{https://ipolcore.ipol.im/demo/clientApp/demo.html?id=77777000095}{https://ipolcore.ipol.im/demo/clientApp/demo.html?id=77777000095}.
\end{ipolSupp}

\ipolKeywords{image forensics, forgery detection}
\newpage

%------------------------------------------------------------------------------
\section{Introduction}

%------------------------------------------------------------------------------
\section{Method}
During demosaicing, missing colours on each pixel are interpolated from its neighbours. As a consequence, pixels that are interpolated in a given channel are more likely to be an intermediate value, in other words, to be neither lower than all its direct neighbours nor higher than all of them. This is especially true with the most simple demosaicing algorithm, such as bilinear demosaicing which interpolate the three channels separately.

This method proposes to count the intermediate values corresponding to each of the four patterns. On the correct pattern, as most pixels are sampled, there should be fewer intermediate values than in the other patterns.

\subsection{Intermediate values detection}
Let $I$ of shape $(X, Y)$ be one colour channel of an image. The pixel at location $(x, y)$ is considered an intermediate value if $\min(I_{x-1, y}, I_{x+1, y}, I_{x, y-1}, I_{x, y+1}) \leq I_{x, y} \leq \max(I_{x-1, y}, I_{x+1, y}, I_{x, y-1}, I_{x, y+1})$.

We define $\mathcal M(I)$ as the mask of intermediate values of $I$. We define it as 1 if $(x, y)$ is an intermediate value of $I$, and $0$ otherwise.

On the border of the image, either $x\pm1$ or $y\pm1$ is out of the image boundaries. To avoid border effects, we would thus have to mask out a 1-pixel border around the image. However, doing this would cause an imbalance in the number of pixels corresponding to different patterns, in other words their would be, in border windows, more pixels corresponding to one pattern than another. To solve the imbalance, we instead mask out a 2-pixels-wide border.

More formally, the mask of intermediate values is thus defined as:
\[\phantom.\mathcal M(I)_{x, y} \triangleq \left\{\begin{array}{ll}
        0 &\text{if }x\in\{0, 1, X-2, X-1\}\text{ or }y\in\{0, 1, Y-2, Y-1\}\\
        1 &\text{otherwise, if}\min(I_{x-1, y}, I_{x+1, y}, I_{x, y-1}, I_{x, y+1}) \leq I_{x, y} \leq \max(I_{x-1, y}, I_{x+1, y}, I_{x, y-1}, I_{x, y+1})\\
0 &\text{otherwise}\end{array}\right.\]
The computation of this mask is described in Algorithm~\ref{alg:intermediate}.

Furthermore, to limit demosaicing artefacts, many demosaicing algorithm tend to avoid interpolating against strong gradients, such as against an edge, and thus often only interpolate in one direction (in which the gradient is smaller). To take this into account, we propose to replace the original isotropic intermediate values mask with bidirectional filters, that separately considers horizontally and vertically intermediate values. We define the mask of horizontal intermediate values as
\[\phantom.\mathcal M(I)^h_{x, y} \triangleq \left\{\begin{array}{ll}
        0 &\text{if }x\in\{0, 1, X-2, X-1\}\text{ or }y\in\{0, 1, Y-2, Y-1\}\\
        1 &\text{otherwise, if}\min(I_{x-1, y}, I_{x+1, y}) \leq I_{x, y} \leq \max(I_{x-1, y}, I_{x+1, y})\\
0 &\text{otherwise}\end{array}\right..\]
Vertical values are computed in a similar way:
\[\phantom.\mathcal M(I)^v_{x, y} \triangleq \left\{\begin{array}{ll}
        0 &\text{if }x\in\{0, 1, X-2, X-1\}\text{ or }y\in\{0, 1, Y-2, Y-1\}\\
        1 &\text{otherwise, if}\min(I_{x, y-1}, I_{x, y+1}) \leq I_{x, y} \leq \max(I_{x, y-1}, I_{x, y+1})\\
0 &\text{otherwise}\end{array}\right..\]
Under this definition, the bidirectional mask of intermediate values is then defined as the mean of the horizontal and vertical masks:
\[\phantom.\mathcal M(I)_{x, y} \triangleq\frac 1 2 \left(M(I)^h_{x, y} + M(I)^v_{x, y}\right).\]
It is thus null at the border and where a pixel is not an intermediate value, equal to $\frac 1 2$ where the pixel is either horizontally or vertically an intermediate value, and equal to $1$ when it is an intermediate value both horizontally and vertically. The computation of the isotropic mask is detailed in Algorithm~\ref{alg:intermediate_bidirectional}.

The original isotropic mask and the bidirectional one will be compared in Section~\ref{sec:experiments}. For the rest of this section, we consider $R$, $G$ and $B$ the masks of intermediate values obtained on the respectively red, green and blue channels of the image. Which of the two methods was used to compute those masks is irrelevant to the rest of the algorithm.

\begin{algorithm}[h]
\caption{Mark intermediate values (original isotropic version)}
\label{alg:intermediate}
\Function{is\_intermediate(arr)}{
\Input{arr}{Array of size $(X, Y)$, one channel of an image}
\Output{mask}{Array of size $(X-4, Y-4)$, intermediate values mask}

mask $\assign \mathbf 0_{(X-4, Y-4)}$\;
\For{$x$ \FROM 2 \TO $X-2$ \AND $y$ \FROM 2 \TO $Y-2$}{
$\mathrm{mi}\assign \min{(\mathrm{arr}_{x+1, y}, \mathrm{arr}_{x, y-1}, \mathrm{arr}_{x-1, y}, \mathrm{arr}_{x, y+1})}$\;
$\mathrm{ma}\assign \max{(\mathrm{arr}_{x+1, y}, \mathrm{arr}_{x, y-1}, \mathrm{arr}_{x-1, y}, \mathrm{arr}_{x, y+1})}$\;
\If{$\mathrm{mi} \leq \mathrm{arr}_{x, y} \leq \mathrm{ma}$}
{
    $\mathrm{mask}_{x-2, y-2}\assign 1$\;
}
}
\Return{mask}
}
\end{algorithm}

\begin{algorithm}[h]
\caption{Mark intermediate values (bidirectional variant)}
\label{alg:intermediate_bidirectional}
\Function{is\_intermediate(arr)}{
\Input{arr}{Array of size $(X, Y)$, one channel of an image}
\Output{mask}{Array of size $(X-4, Y-4)$, intermediate values mask}

mask $\assign \mathbf 0_{(X-4, Y-4)}$\;
\For{$x$ \FROM 2 \TO $X-2$ \AND $y$ \FROM 2 \TO $Y-2$}{
        $\mathrm{mi}_h\assign \min(\mathrm{arr}_{x-1, y}, \mathrm{arr}_{x+1, y})$\;
        $\mathrm{ma}_h\assign \max(\mathrm{arr}_{x-1, y}, \mathrm{arr}_{x+1, y})$\;
        $\mathrm{mi}_v\assign \min(\mathrm{arr}_{x, y-1}, \mathrm{arr}_{x, y+1})$\;
        $\mathrm{ma}_v\assign \max(\mathrm{arr}_{x, y-1}, \mathrm{arr}_{x, y+1})$\;

\If{$\mathrm{mi_h} \leq \mathrm{arr}_{x, y} \leq \mathrm{ma_h}$}
{
    $\mathrm{mask}_{x-2, y-2}\peq \frac 1 2$\;
}
\If{$\mathrm{mi_v} \leq \mathrm{arr}_{x, y} \leq \mathrm{ma_v}$}
{
    $\mathrm{mask}_{x-2, y-2}\peq \frac 1 2$\;
}
}
\Return{mask}
}
\end{algorithm}

\subsection{Division into windows}
The strategy to find forgeries using inconsistencies in the CFA patterns is to first find in which pattern the full image has been demosaiced, then to find the pattern used in different windows of the image. If a the pattern detected in a window is different than the one detected for the full image, then this window is inconsistent with the rest of the image and can be considered as forged.

To improve the precision of detection, we do not simply use adjacent windows, but rather sliding windows with an overlap. The window size $W$ and stride are set as parameters of the algorithm. The stride determines the number of pixel between the left (or top) border of two consecutive windows, such as a stride equal to the window size leads to adjacent windows without overlapping, a stride equal to half the window size leads to a new window starting at the middle of the previous one, etc.

Using a lower stride will not drastically improve the detection, but may help contour a detected forgery more precisely, at the cost of a slower algorithm.

In the rest of this section, we consider the windowed intermediate values as of shape $(X_w, Y_w, 3, W, W)$, where $X_w$ and $Y_w$ are the number of windows in a row and in a column. In practice, the windows are computed in one dimension, and the full array is thus of shape $(X_w*Y_w, 3, W, W)$, and is only reshaped at the end of the computation.

\subsection{Finding the pattern}
The four patterns can be divided into two subgroups by their diagonal: \textsc{rggb} and \textsc{bggr} share the \textsc{·gg·} diagonal, whereas \textsc{grbg} and \textsc{gbrg} share the \textsc{g··g} diagonal. Because the Bayer CFA samples twice as many pixels in green than in red or blue, it is easier to find information on the pattern in the green channel. This is amplified by the fact that many demosaicing algorithms first interpolate the green channel by itself, but interpolate the red and blue channels using information from the green channel.

As a consequence, the presented method first tries to detect the diagonal pattern using the green channel(\textsc{·gg·} or \textsc{g··g}), then uses the red and blue channels to compare the two potential patterns sharing that diagonal.

Let $R$, $G$ and $B$ be the masks of intermediate values on the respectively red, green and blue channels, These masks, each of shape $(2X, 2Y)$, can represent either the full image or a window of it. To maintain the balance between patterns, those masks must be of even size. For this reasons, the window size must be even, and the last row/column of the full image is removed if necessary to ensure the evenness of the shape

We start by looking at the green channel for the diagonal grids. The intermediate value count corresponding to the \textsc{·gg·} pattern is
\[C_{\textsc{·gg·}} \triangleq \sum_{x=0}^X\sum_{y=0}^Y\left(G_{2x+1, 2y} + G_{2x, 2y+1}\right)\]
while the count corresponding to the \textsc{g··g} pattern is
\[\phantom.C_{\textsc{g··g}} \triangleq \sum_{x=0}^X\sum_{y=0}^Y\left(G_{2x, 2y} + G_{2x+1, 2y+1}\right).\]

The count difference of the diagonal is then defined as
\[\phantom.\Delta_{\mathrm{diag}}  \triangleq \frac 1 {2X\cdot Y} \left(C_{\textsc{·gg·}} - C_{\textsc{g··g}}\right)\]
This difference is positive if the detected diagonal is \textsc{g··g}, and negative if it is \textsc{·gg·}:
\[\phantom.D\triangleq\left\{\begin{array}{lc}\textsc{g··g}&\Delta_{\mathrm{diag}}>0\\\textsc{·gg·}&\Delta_{\mathrm{diag}}>0\\-1&\text{otherwise}\end{array}\right..\]
        The normalization by $\frac{1}{2XY}$ means that the resulting difference is in $[-1, 1]$, and is equal to $\pm1$ if all pixels in one of the patterns are intermediate values, whereas the other pattern has no intermediate values--~$XY$ is the number of $2\times2$ blocks in a mask of shape $(2X, 2Y)$, and we sum two pixels in this block for each pattern. Note that the $\pm1$ limit is only theoretical: Even with bilinear demosaicing, where all interpolated pixels are intermediate values, sampled pixels can be intermediate too, eg.\@ in a slope. As a consequence, the difference will not reach those values in natural cases.

Once we know the main diagonal, we can compare the two patterns sharing that diagonal. The green channel does not provide any information on this, so we use the red and blue channels.

The count of intermediate values corresponding to each pattern is:
\[\phantom.\begin{array}{ll}
        C_{\textsc{rggb}} &\triangleq \sum_{x=0}^X\sum_{y=0}^Y\left(R_{2x, 2y} + B_{2x+1, 2y+1}\right)\\
        C_{\textsc{bggr}} &\triangleq \sum_{x=0}^X\sum_{y=0}^Y\left(R_{2x+1, 2y+1} + B_{2x, 2y}\right)\\
        C_{\textsc{grbg}} &\triangleq \sum_{x=0}^X\sum_{y=0}^Y\left(R_{2x+1, 2y} + B_{2x, 2y+1}\right)\\
        C_{\textsc{gbrg}} &\triangleq \sum_{x=0}^X\sum_{y=0}^Y\left(R_{2x, 2y+1} + B_{2x+1, 2y}\right)
\end{array}.\]

The count differences of the two pattern pairs are then defined as
\[\begin{array}{ll}
        \Delta_{\textsc{rggb}-\textsc{bggr}} &\triangleq \frac 1 {2XY}\left( C_{\textsc{rggb}} - C_{\textsc{bggr}}\right)\\
\Delta_{\textsc{grbg}-\textsc{gbrg}} &\triangleq \frac 1 {2XY}\left(C_{\textsc{rggb}} - C_{\textsc{gbrg}}\right)\end{array}\]
and are then combined into the main grid difference
\[\phantom.\Delta_{\mathrm{main}}\triangleq\left\{\begin{array}{lc}
        \Delta_{\textsc{rggb}-\textsc{bggr}}&D=\textsc{g··g}\\
\Delta_{\textsc{grbg}-\textsc{gbrg}}&D=\textsc{·gg·}\end{array}\right..\]
Finally, the main detected grid can be obtained:
\[\phantom.M \triangleq \left\{\begin{array}{lc}
        \textsc{rggb}&D=\textsc{·gg·}\text{ and }\Delta_{\mathrm{main}}<0\\
        \textsc{bggr}&D=\textsc{·gg·}\text{ and }\Delta_{\mathrm{main}}>0\\
        \textsc{grbg}&D=\textsc{g··g}\text{ and }\Delta_{\mathrm{main}}<0\\
        \textsc{gbrg}&D=\textsc{g··g}\text{ and }\Delta_{\mathrm{main}}>0\\
-1&\text{otherwise}\end{array}\right..\]

Both for the diagonal and main grids, if there is strict equality in the two counts detected, no grid is considered detected. Naturally, if no decision is taken on the diagonal, no main grid is selected either.

The grid detection is detailed in Algorithm~\ref{alg:grid}.

While $\Delta_{\mathrm{main}}$ is later used to make the decisions on forgeries, the two intermediary comparisons $\Delta_{\textsc{rggb}-\textsc{bggr}}$ and $\Delta_{\textsc{grbg}-\textsc{gbrg}}$ are easier to understand visually, and are thus kept for visualization.

Our implementation of the count difference computation is slightly different from the description of the original article. In the original article, the difference is not normalised by $\frac 1 {2XY}$. More importantly, the difference is computed separately in the red and blue channels, and the strongest of the two is kept, whereas we use their sum.
The reason for this is that the original article only tries to classify in which pattern an image has been sampled, without considering how confident one can be in the detection, or how to use it to detect forgeries. When only considering classification of an image or window into the four patterns, both the original article and our implementation provide the same results. However, adding the normalization and summing the two channels makes it easier for us to also compute a confidence value for the detections, which will be described in the next subsection.

Finally, we note that even though this algorithm is presented for one window, the grid detection is in practice done on all windows simultaneously.

\begin{algorithm}[h]
\caption{Find the grid}
\label{alg:grid}
\Function{find\_grid(R, G, B)}
{
        \qb{TBD: reorder the final if blocks to follow what is explained and done (even though this is equivalent).}
\Input{R}{Array of even size $(2X, 2Y)$, typically as returned by \FuncSty{is\_intermediate} or a sub-window of it on the red channel}
\Input{G}{Same as above for the green channel}
\Input{B}{Same as above for the blue channel}
\Output{main}{CFA pattern identified by the function (one of \textsc{rggb}, \textsc{grbg}, \textsc{gbrg}, \textsc{bggr})}
\Output{diag}{Diagonal pattern identified by the function (either \textsc{·gg·} or \textsc{g··g})}
\Output{diff\_main}{Difference of count in intermediate values between the two pattern sharing the same diagonal. Positive if the best pattern is \textsc{rggb} or \textsc{grbg}, negative if the best pattern is \textsc{gbrg} or \textsc{bggr}.}
\Output{diff\_diag}{Difference of count of intermediate values between the two diagonal patterns.}
\Comment{First we select the best diagonal pattern using the green values}
        $\mathrm{count_{*GG*}}\assign \sum_{x=0}^X\sum_{y=0}^YG_{2x,2y+1} + G_{2x+1, 2y}$\;
        $\mathrm{count_{G**G}}\assign \sum_{x=0}^X\sum_{y=0}^YG_{2x,2y} + G_{2x+1, 2y+1}$\;
$\mathrm{diff\_diag} \assign \frac 1{2XY}\left(\mathrm{count_{*GG*}} - \mathrm{count_{G**G}}\right)$\;
\If{$\mathrm{diff\_diag}$ < 0}{
diag$\assign$ \textsc{·gg·}\;
}
\Else{
diag$\assign$ \textsc{g··g}\;
}

\Comment{Now we select within the two patterns sharing the detected diagonal.}
\If{diff\_diag$=$\textsc{·gg·}}
{
\Comment{Either \textsc{rggb} or \textsc{bggr}}
        $\mathrm{count_{RGGB}}\assign \sum_{x=0}^{X}\sum_{y=0}^YR_{2x,2y} + B_{2x+1, 2y+1}$\;
        $\mathrm{count_{BGGR}}\assign \sum_{x=0}^X\sum_{y=0}^Y R_{2x+1,2y+1} + B_{2x, 2y}$\;
$\mathrm{diff\_main} \assign \frac 1{2XY}\left(\mathrm{count_{RGGB}} - \mathrm{count_{BGGR}}\right)$\;
\If{$\mathrm{diff\_main}$ < 0}{
main$\assign$ \textsc{rggb}\;
}
\Else{
main$\assign$ \textsc{bggr}\;
}
}
\Else
{
\Comment{Either \textsc{grbg} or \textsc{gbrg}}
$\mathrm{count_{GRBG}}\assign \sum_{x=0}^{\frac X 2}R_{2x+1,2y} + B_{2x, 2y+1}$\;
$\mathrm{count_{GBRG}}\assign \sum_{x=0}^{\frac X 2}R_{2x,2y+1} + B_{2x+1, 2y}$\;
$\mathrm{diff\_main} \assign \frac 2{XY}\left(\mathrm{count_{GRBG}} - \mathrm{count_{GBRG}}\right)$\;
\If{$\mathrm{diff\_main}$ < 0}{
main$\assign$ \textsc{grbg}\;
}
\Else{
main$\assign$ \textsc{gbrg}\;
}
}
\Return{main, diag, diff\_main, diff\_diag}
}
\end{algorithm}

\subsection{Forgery detection}
Using the previously-described algorithms, we can compute the intermediate value masks in all channels, cut them into windows, and detect the diagonal and pattern of the global image and of each window.

With this information, we could simply say that the windows which do not use the same pattern than the main grid correspond to forged regions. However, doing this creates many false positives, as the detection is not always correct.

We do not propose here a way to automatically decide, without arbitrary parameters, which region to detect as forged. However, we show that a simple thresholding can filter out most of the false, noise-like positives, while keeping the relevant detections.

We segment the windows into connected components by their grids. In other words, a connected component is a set of spatially connected windows whose detected pattern is the same.
This segmentation is done with \texttt{scikit-image}~\cite{skimage}.

Components whose detected pattern is the global image's are immediately discarded; they are not considered forged as they are coherent with the full image.
For components whose detected pattern is different, we consider them as forged, with a confidence value which corresponds to the maximum absolute difference of count of all windows in that components (either $|\Delta_{\mathrm{main}}|$ or $|\Delta_{\mathrm{diag}}|$ depending on whether we are looking at the full pattern or the diagonal). In other words, the confidence of a component is that of the most prominent window it contains.

We thus have two confidence maps: one for the diagonal and one for the full pattern. Those two are merged into a final detection map by taking its local maximum.

If someone needed a binary output map, thresholding this map with a threshold $\gamma$ would amount to performing grid-by-grid hysteresis thresholding on the original confidence map, with a lower threshold 0 and a higher threshold $\gamma$.

We noted earlier that the maximal value of the difference, $\pm1$, was only theoretical and would not be reached in natural cases. Even with simple algorithms, this value rarely goes out of $[-0.3, 0.3]$. As a consequence, a region detected with a confidence region of, for instance, 0.2, should already be considered as detected with high confidence.

We do not normalize it further for the demo. However, when the detection map needs to be normalized to $[0, 1]$, for instance when computing metrics on the results, we propose to clip the output to $[0, \gamma]$, then to divide the result by $\gamma$ to have an output in the desired range. Our experiments show that, on the Trace~\cite{trace} dataset used, a value of $\gamma=X$\qb{provide value} provided good results. This is, however, not a claim that this threshold is optimal for any usage. The choice of the threshold is left to the user.




\begin{algorithm}[h]
        \caption{Global algorithm}
        \label{alg:global}
        \Function{find\_forgeries(img, W, stride, threshold)}
        {
                \Input{img}{Input image, size $(X, Y, 3)$}
                \Parameter{W}{int, Window size}
                \Parameter{stride}{int, Distance between the left/top border of two consecutive windows. Must divide W. If equal to it, windows will be adjacent without overlapping.}
                \Parameter{threshold}{float, higher hysteresis threshold to select relevant inconsistencies.}
                \Output{main, diag, diff\_main, diff\_diag}{Windowed output of \texttt{find\_grid}. See Alg.~\ref{alg:grid} for more details.}
                \Output{forged\_main, forged\_diag}{Windows whose detected / diagonal pattern is inconsistent with the global pattern}
                \Output{forged\_main\_thresholded, forged\_diag\_thresholded}{Same, after hysteresis thresholding}
                \Output{coords$_x$, coords$_y$}{Coordinates corresponding to the center of each window.}
                \Comment{Crop the image if needed, as $X$ and $Y$ need to be even.}
                $\mathrm{img} \assign \mathrm{img}[:X-X\%2, :Y-Y\%2]$\;
                $\mathrm{intermediate} \assign \mathtt{is\_intermediate}(\mathrm{img})$\;
                $\mathrm{windows}, \mathrm{coords_x}, \mathrm{coords_y} \assign \mathtt{get\_windows}(\mathrm{intermediate}, W, \mathrm{stride})$\;
                \Comment{Correct the coordinates to account for the lost border from \texttt{is\_intermediate}.}
                $\mathrm{coords_x}, \mathrm{coords_y} \peq 2$\;
                \Comment{Number of window rows and columns.}
                $X_w, Y_w \assign |\mathrm{coords_x}|, |\mathrm{coords_y}|$\;
                $\mathrm{global\_main}, \mathrm{global\_diag}, \_, \_ = \mathtt{find\_grid}(\mathrm{intermediate}[:, :, 0], \mathrm{intermediate}[:, :, 1], \mathrm{intermediate}[:, :, 2])$\;
                $\mathrm{main}, \mathrm{diag}, \mathrm{diff\_main}, \mathrm{diff\_diag} \assign \mathbf{0}_{X_w, Y_w}$\;
                \For{$x$ \FROM 0 \TO $X_w$ \AND $y$ \FROM 0 \TO $Y_w$}{
                        $\mathrm{main}[x, y], \mathrm{diag}[x, y], \mathrm{diff\_main}[x, y], \mathrm{diff\_diag}[x, y] \assign \mathtt{find\_grid(\mathrm{windows}[x, y, 0], \mathrm{windows}[x, y, 1], \mathrm{windows}[x, y, 2])}$\;
                }
                \Comment{Find inconsistent regions}
                $\mathrm{forged\_diag} \assign \mathrm{diag}\neq\mathrm{global\_diag}$\;
                $\mathrm{forged\_main} \assign \mathrm{main}\neq\mathrm{global\_main}$\;
                \Comment{Soft values}
                $\mathrm{labels\_main} \assign \mathtt{label_connected}(\mathrm{diag, global\_diag})$\;
                $\mathrm{forged\_main\_soft}\assign\mathbf0_{X_W, Y_W}$\;
                \For{$\mathrm{label}$ \FROM 0 \TO $\max(\mathrm{labels\_main})$}
                {
                        $\mathrm{forged\_main\_soft}\peq \max\left((\mathrm{labels\_main}=\mathrm{label})\odot|\mathrm{diff\_main}|\right)$\;
                }
                $\mathrm{labels\_diag} \assign \mathtt{label_connected}(\mathrm{diag, global\_diag})$\;
                $\mathrm{forged\_diag\_soft}\assign\mathbf0_{X_W, Y_W}$\;
                \For{$\mathrm{label}$ \FROM 0 \TO $\max(\mathrm{labels\_diag})$}
                {
                        $\mathrm{forged\_diag\_soft}\peq \max\left((\mathrm{labels\_diag}=\mathrm{label})\odot|\mathrm{diff\_diag}|\right)$\;
                }
                
                

                %$\mathrm{forged\_main\_thresholded}  \assign \mathbf 0_{X_w, Y_w}$\;
                %\For{$g\in(\textsc{rggb}, \textsc{grbg}, \textsc{gbrg}, \textsc{bggr})$}
                %{
                        %\If{$g\neq\mathrm{global_main}$}
                        %{
                                %\Comment{Absolute difference where the grid is $g$, 0 elsewhere}
                                %$\mathrm{values} \assign |\mathrm{diff\_main}|\odot(\mathrm{grid}=g)$\;
                                %$\mathrm{forged\_main\_thresholded} \peq \mathtt{apply\_hysteresis\_threshold}(\mathrm{values}, 0, \mathrm{threshold})$\;
                        %}
                %}
                %\Comment{Same on the diagonals}
                %$\mathrm{values} \assign |\mathrm{diff\_diag}|\odot(\mathrm{diag}\neq\mathrm{global\_diag})$\;
                %$\mathrm{forged\_diag\_thresholded} \assign \mathtt{apply\_hysteresis\_threshold}(\mathrm{values}, 0, \mathrm{threshold})$\;
                \Return{main, diag, forged\_main, forged\_diag, forged\_main\_soft, forged\_diag\_soft, diff\_main, diff\_diag, coords$_x$, coords$_y$}
        }
\end{algorithm}


\clearpage


\iffalse

%------------------------------------------------------------------------------
\section{Experiments}
To evaluate the ability of this method to detect the CFA pattern correctly, we take 15 images from the Raise Dataset~\cite{raise}, and demosaick them using the 7 algorithms available in LibRaw: Bilinear interpolation, AAHD, AHD, DCB, DHT, PPG and VNG. 11 of these images are of size $4948\times3280$, the other 4 are of size $4310\times2868$. The selected images can be seen in Fig.~\ref{fig:15images}.

\begin{figure}[ht]
    \centering
    \begin{subfigure}[c]{.31\linewidth}\centering
    \includegraphics[height=\linewidth]{images/original/r002fc3e2t.jpeg}
    \caption{r002fc3e2t}
    \end{subfigure}\hfill%
    \begin{subfigure}[c]{.31\linewidth}\centering
    \includegraphics[height=\linewidth]{images/original/r1ead3024t.jpeg}
    \caption{r1ead3024t}
    \end{subfigure}\hfill%
    \begin{subfigure}[c]{.31\linewidth}\centering
    \includegraphics[height=\linewidth]{images/original/r1ceba29dt.jpeg}
    \caption{r1ceba29dt}
    \end{subfigure}%
    
    \begin{subfigure}[c]{.31\linewidth}\centering
    \includegraphics[width=\linewidth]{images/original/r0a2ff882t.jpeg}
    \caption{r0a2ff882t}
    \end{subfigure}\hfill%
    \begin{subfigure}[c]{.31\linewidth}\centering
    \includegraphics[width=\linewidth]{images/original/r0a808003t.jpeg}
    \caption{r0a808003t}
    \end{subfigure}\hfill%
    \begin{subfigure}[c]{.31\linewidth}\centering
    \includegraphics[width=\linewidth]{images/original/r0a966704t.jpeg}
    \caption{r0a966704t}
    \end{subfigure}
    
    \begin{subfigure}[c]{.31\linewidth}\centering
    \includegraphics[width=\linewidth]{images/original/r0e04cc91t.jpeg}
    \caption{r0e04cc91t}
    \end{subfigure}\hfill%
    \begin{subfigure}[c]{.31\linewidth}\centering
    \includegraphics[width=\linewidth]{images/original/r0ea0825ft.jpeg}
    \caption{r0ea0825ft}
    \end{subfigure}\hfill%
    \begin{subfigure}[c]{.31\linewidth}\centering
    \includegraphics[width=\linewidth]{images/original/r1a0f5585t.jpeg}
    \caption{r1a0f5585t}
    \end{subfigure}
    
    \begin{subfigure}[c]{.31\linewidth}\centering
    \includegraphics[width=\linewidth]{images/original/r1c9fdcf4t.jpeg}
    \caption{r1c9fdcf4t}
    \end{subfigure}\hfill%
    \begin{subfigure}[c]{.31\linewidth}\centering
    \includegraphics[width=\linewidth]{images/original/r06aa7dabt.jpeg}
    \caption{r06aa7dabt}
    \end{subfigure}\hfill%
    \begin{subfigure}[c]{.31\linewidth}\centering
    \includegraphics[width=\linewidth]{images/original/r07cfb432t.jpeg}
    \caption{r07cfb432t}
    \end{subfigure}
    
    \begin{subfigure}[c]{.31\linewidth}\centering
    \includegraphics[width=\linewidth]{images/original/r07ffdc87t.jpeg}
    \caption{r07ffdc87t}
    \end{subfigure}\hfill%
    \begin{subfigure}[c]{.31\linewidth}\centering
    \includegraphics[width=\linewidth]{images/original/r16da5576t.jpeg}
    \caption{r16da5576t}
    \end{subfigure}\hfill%
    \begin{subfigure}[c]{.31\linewidth}\centering
    \includegraphics[width=\linewidth]{images/original/r191f3cdet.jpeg}
    \caption{r191f3cdet}
    \end{subfigure}
    
    \caption{Those 15 images from the Raise Dataset~\cite{raise} were used during our experiments.}
    \label{fig:15images}
\end{figure}

\subsection{CFA pattern detection}
\begin{table}[ht]
    \centering
    \begin{tabular}{lcc}
    \toprule
    Demosaicking & Diagonal & Full pattern\\
    \midrule
    AAHD & \color{c2}0/15 & \color{c2}0/15\\
    AHD & \color{c0}15/15 & \color{c0}15/15\\
    DCB & \color{c0}15/15 & \color{c0}15/15\\
    DHT & \color{c2}3/15 & \color{c2}3/15\\
    Bilinear & \color{c0}15/15 & \color{c0}15/15\\
    PPG & \color{c0}15/15 & \color{c3}13/15\\
    VNG & \color{c0}15/15 & \color{c3}14/15\\
    \bottomrule
    \end{tabular}
    \caption{Identification of the main diagonal and of the full pattern on the 15 images. The algorithm works very well when the demosaicking is done with AHD, DCB or Bilinear demosaicking, with a few errors on the full pattern against PPG- or VNG-demosaicked images. It fails to detect even the diagonal on AAHD- and DHT-demosaicked images}
    \label{tab:global}
\end{table}

\def\s{.12\linewidth}
\setlength{\tabcolsep}{0.2em}
\begin{figure}[ht]
        \centering
        \begin{tabular}{cccccccc}
                \multicolumn{8}{c}{
                        \begin{tabular}{cl}
                                \includegraphics[width=.6\linewidth]{images/original/r0a2ff882t.jpeg}&
                                \includegraphics[height=150pt]{images/cb.png}\\
                                Original image (r0a2ff882t), in \textsc{rggb} pattern&
                        \end{tabular}
                }\\                                
                & AAHD & AHD & DCB & DHT & Bilinear & PPG & VNG\\
                \midrule
                \raisebox{5pt}{\rotatebox{90}{\tiny Original}} & 
                \includegraphics[width=\s]{images/bike/AAHD/iso_64_grids.png} &
                \includegraphics[width=\s]{images/bike/AHD/iso_64_grids.png} &
                \includegraphics[width=\s]{images/bike/DCB/iso_64_grids.png} &
                \includegraphics[width=\s]{images/bike/DHT/iso_64_grids.png} &
                \includegraphics[width=\s]{images/bike/LINEAR/iso_64_grids.png} &
                \includegraphics[width=\s]{images/bike/PPG/iso_64_grids.png} &
                \includegraphics[width=\s]{images/bike/VNG/iso_64_grids.png} \\
                \rotatebox{90}{\tiny Bidirectional} & 
                \includegraphics[width=\s]{images/bike/AAHD/bid_64_grids.png} &
                \includegraphics[width=\s]{images/bike/AHD/bid_64_grids.png} &
                \includegraphics[width=\s]{images/bike/DCB/bid_64_grids.png} &
                \includegraphics[width=\s]{images/bike/DHT/bid_64_grids.png} &
                \includegraphics[width=\s]{images/bike/LINEAR/bid_64_grids.png} &
                \includegraphics[width=\s]{images/bike/PPG/bid_64_grids.png} &
                \includegraphics[width=\s]{images/bike/VNG/bid_64_grids.png}\\
                \bottomrule
        \end{tabular}
\caption{Results of the method on 64×64 windows, both with the original isotropic intermediate value mask and the proposed bidirectional one, on one image with the 7 different demosaicing algorithms. Both method work perfectly on the DCB- and Bilinear-demosaiced images. With the AHD, PPG and VNG methods, both methods have trouble discerning between the two patterns sharing the same diagonal, but the bidirectional detection makes fewer mistakes. Textured regions such as the basket can create a localized shift in the detected mosaic, which could be interpretated as a forgery.
With the AAHD and DHT algorithm, the method consistently detects the wrong diagonal.}
\label{fig:bike}
\end{figure}



\begin{figure}[ht]
\centering
\begin{subfigure}[t]{.5\linewidth}
\includegraphics[width=\linewidth]{images/bike/ahd_iso_64_diff_rggb_bggr.png}
\caption{Isotropic}
\end{subfigure}%
\begin{subfigure}[t]{.5\linewidth}
\includegraphics[width=\linewidth]{images/bike/ahd_bid_64_diff_rggb_bggr.png}
\caption{Bidirectional}
\end{subfigure}%
\caption{This figure shows, on the AHD-demosaiced bicycle image, the difference of counts of intermediate values corresponding to the \textsc{rggb} and \textsc{bggr} patterns, on the red and blue channels. This count is what is used by the algorithm to decide on a grid. A negative difference corresponds to the correct \textsc{rggb} pattern, a positive difference to the incorrect \textsc{bggr} pattern. The difference is normalized by dividing it by the size of the block ($64\times64$). The textures in the basket area leads to a locally consistent shift in the position of the intermediate values. The error is slightly less prominent when a bidirectional mask is used, but is still consistently in favour of the wrong grid.}
\end{figure}

\begin{figure}[ht]
        \centering
        \begin{subfigure}[t]{\linewidth}
        \begin{tabular}{ccccccccc}
                \multicolumn{9}{c}{
                        \begin{tabular}{cl}
                                \includegraphics[height=150pt]{images/original/r07ffdc87t.jpeg}&
                                \includegraphics[height=150pt]{images/cb.png}
                        \end{tabular}
                }\\                                
                && AAHD & AHD & DCB & DHT & Bilinear & PPG & VNG\\
                \midrule
                \multirow{2}{*}[1.3em]{{\rotatebox[origin=c]{90}{Uncompressed}}}&
                \raisebox{5pt}{\rotatebox{90}{\tiny Original}} & 
                \includegraphics[width=\s]{images/carnival/AAHD/iso_64_grids.png}&
                \includegraphics[width=\s]{images/carnival/AHD/iso_64_grids.png}&
                \includegraphics[width=\s]{images/carnival/DCB/iso_64_grids.png}&
                \includegraphics[width=\s]{images/carnival/DHT/iso_64_grids.png}&
                \includegraphics[width=\s]{images/carnival/LINEAR/iso_64_grids.png}&
                \includegraphics[width=\s]{images/carnival/PPG/iso_64_grids.png}&
                \includegraphics[width=\s]{images/carnival/VNG/iso_64_grids.png}\\
                &\rotatebox{90}{\tiny Bidirectional}&
                \includegraphics[width=\s]{images/carnival/AAHD/bid_64_grids.png}&
                \includegraphics[width=\s]{images/carnival/AHD/bid_64_grids.png}&
                \includegraphics[width=\s]{images/carnival/DCB/bid_64_grids.png}&
                \includegraphics[width=\s]{images/carnival/DHT/bid_64_grids.png}&
                \includegraphics[width=\s]{images/carnival/LINEAR/bid_64_grids.png}&
                \includegraphics[width=\s]{images/carnival/PPG/bid_64_grids.png}&
                \includegraphics[width=\s]{images/carnival/VNG/bid_64_grids.png}\\
                \cmidrule{1-2}
                \multirow{2}{*}[.5em]{{\rotatebox[origin=c]{90}{JPEG 100}}}&
                \raisebox{5pt}{\rotatebox{90}{\tiny Original}} & 
                \includegraphics[width=\s]{images/carnival/AAHD/iso_j100_64_grids.png}&
                \includegraphics[width=\s]{images/carnival/AHD/iso_j100_64_grids.png}&
                \includegraphics[width=\s]{images/carnival/DCB/iso_j100_64_grids.png}&
                \includegraphics[width=\s]{images/carnival/DHT/iso_j100_64_grids.png}&
                \includegraphics[width=\s]{images/carnival/LINEAR/iso_j100_64_grids.png}&
                \includegraphics[width=\s]{images/carnival/PPG/iso_j100_64_grids.png}&
                \includegraphics[width=\s]{images/carnival/VNG/iso_j100_64_grids.png}\\
                &\rotatebox{90}{\tiny Bidirectional}&
                \includegraphics[width=\s]{images/carnival/AAHD/bid_j100_64_grids.png}&
                \includegraphics[width=\s]{images/carnival/AHD/bid_j100_64_grids.png}&
                \includegraphics[width=\s]{images/carnival/DCB/bid_j100_64_grids.png}&
                \includegraphics[width=\s]{images/carnival/DHT/bid_j100_64_grids.png}&
                \includegraphics[width=\s]{images/carnival/LINEAR/bid_j100_64_grids.png}&
                \includegraphics[width=\s]{images/carnival/PPG/bid_j100_64_grids.png}&
                \includegraphics[width=\s]{images/carnival/VNG/bid_j100_64_grids.png}\\
                \cmidrule{1-2}
                \multirow{2}{*}[.5em]{{\rotatebox[origin=c]{90}{JPEG 98}}}&
                \raisebox{5pt}{\rotatebox{90}{\tiny Original}} & 
                \includegraphics[width=\s]{images/carnival/AAHD/iso_j98_64_grids.png}&
                \includegraphics[width=\s]{images/carnival/AHD/iso_j98_64_grids.png}&
                \includegraphics[width=\s]{images/carnival/DCB/iso_j98_64_grids.png}&
                \includegraphics[width=\s]{images/carnival/DHT/iso_j98_64_grids.png}&
                \includegraphics[width=\s]{images/carnival/LINEAR/iso_j98_64_grids.png}&
                \includegraphics[width=\s]{images/carnival/PPG/iso_j98_64_grids.png}&
                \includegraphics[width=\s]{images/carnival/VNG/iso_j98_64_grids.png}\\
                &\rotatebox{90}{\tiny Bidirectional}&
                \includegraphics[width=\s]{images/carnival/AAHD/bid_j98_64_grids.png}&
                \includegraphics[width=\s]{images/carnival/AHD/bid_j98_64_grids.png}&
                \includegraphics[width=\s]{images/carnival/DCB/bid_j98_64_grids.png}&
                \includegraphics[width=\s]{images/carnival/DHT/bid_j98_64_grids.png}&
                \includegraphics[width=\s]{images/carnival/LINEAR/bid_j98_64_grids.png}&
                \includegraphics[width=\s]{images/carnival/PPG/bid_j98_64_grids.png}&
                \includegraphics[width=\s]{images/carnival/VNG/bid_j98_64_grids.png}\\
                \cmidrule{1-2}
                \multirow{2}{*}[.5em]{{\rotatebox[origin=c]{90}{JPEG 95}}}&
                \raisebox{5pt}{\rotatebox{90}{\tiny Original}} & 
                \includegraphics[width=\s]{images/carnival/AAHD/iso_j95_64_grids.png}&
                \includegraphics[width=\s]{images/carnival/AHD/iso_j95_64_grids.png}&
                \includegraphics[width=\s]{images/carnival/DCB/iso_j95_64_grids.png}&
                \includegraphics[width=\s]{images/carnival/DHT/iso_j95_64_grids.png}&
                \includegraphics[width=\s]{images/carnival/LINEAR/iso_j95_64_grids.png}&
                \includegraphics[width=\s]{images/carnival/PPG/iso_j95_64_grids.png}&
                \includegraphics[width=\s]{images/carnival/VNG/iso_j95_64_grids.png}\\
                &\rotatebox{90}{\tiny Bidirectional}&
                \includegraphics[width=\s]{images/carnival/AAHD/bid_j95_64_grids.png}&
                \includegraphics[width=\s]{images/carnival/AHD/bid_j95_64_grids.png}&
                \includegraphics[width=\s]{images/carnival/DCB/bid_j95_64_grids.png}&
                \includegraphics[width=\s]{images/carnival/DHT/bid_j95_64_grids.png}&
                \includegraphics[width=\s]{images/carnival/LINEAR/bid_j95_64_grids.png}&
                \includegraphics[width=\s]{images/carnival/PPG/bid_j95_64_grids.png}&
                \includegraphics[width=\s]{images/carnival/VNG/bid_j95_64_grids.png}\\
                \bottomrule
        \end{tabular}
        \caption{Image r07ffdc87t, in \textsc{rggb} pattern}
\end{subfigure}
\end{figure}

\begin{figure}[ht]
          \ContinuedFloat
        \centering
        
        \begin{subfigure}[t]{\linewidth}
        \begin{tabular}{ccccccccc}
                \multicolumn{9}{c}{
                        \begin{tabular}{cl}
                                \includegraphics[height=150pt]{images/original/r0ea0825ft.jpeg}&
                                \includegraphics[height=150pt]{images/cb.png}
                                \\
                        \end{tabular}
                }\\                                
                && AAHD & AHD & DCB & DHT & Bilinear & PPG & VNG\\
                \midrule
                \multirow{2}{*}[1.3em]{{\rotatebox[origin=c]{90}{Uncompressed}}}&
                \raisebox{5pt}{\rotatebox{90}{\tiny Original}} & 
                \includegraphics[width=\s]{images/night/AAHD/iso_64_grids.png}&
                \includegraphics[width=\s]{images/night/AHD/iso_64_grids.png}&
                \includegraphics[width=\s]{images/night/DCB/iso_64_grids.png}&
                \includegraphics[width=\s]{images/night/DHT/iso_64_grids.png}&
                \includegraphics[width=\s]{images/night/LINEAR/iso_64_grids.png}&
                \includegraphics[width=\s]{images/night/PPG/iso_64_grids.png}&
                \includegraphics[width=\s]{images/night/VNG/iso_64_grids.png}\\
                &\rotatebox{90}{\tiny Bidirectional}&
                \includegraphics[width=\s]{images/night/AAHD/bid_64_grids.png}&
                \includegraphics[width=\s]{images/night/AHD/bid_64_grids.png}&
                \includegraphics[width=\s]{images/night/DCB/bid_64_grids.png}&
                \includegraphics[width=\s]{images/night/DHT/bid_64_grids.png}&
                \includegraphics[width=\s]{images/night/LINEAR/bid_64_grids.png}&
                \includegraphics[width=\s]{images/night/PPG/bid_64_grids.png}&
                \includegraphics[width=\s]{images/night/VNG/bid_64_grids.png}\\
                \cmidrule{1-2}
                \multirow{2}{*}[.5em]{{\rotatebox[origin=c]{90}{JPEG 100}}}&
                \raisebox{5pt}{\rotatebox{90}{\tiny Original}} & 
                \includegraphics[width=\s]{images/night/AAHD/iso_j100_64_grids.png}&
                \includegraphics[width=\s]{images/night/AHD/iso_j100_64_grids.png}&
                \includegraphics[width=\s]{images/night/DCB/iso_j100_64_grids.png}&
                \includegraphics[width=\s]{images/night/DHT/iso_j100_64_grids.png}&
                \includegraphics[width=\s]{images/night/LINEAR/iso_j100_64_grids.png}&
                \includegraphics[width=\s]{images/night/PPG/iso_j100_64_grids.png}&
                \includegraphics[width=\s]{images/night/VNG/iso_j100_64_grids.png}\\
                &\rotatebox{90}{\tiny Bidirectional}&
                \includegraphics[width=\s]{images/night/AAHD/bid_j100_64_grids.png}&
                \includegraphics[width=\s]{images/night/AHD/bid_j100_64_grids.png}&
                \includegraphics[width=\s]{images/night/DCB/bid_j100_64_grids.png}&
                \includegraphics[width=\s]{images/night/DHT/bid_j100_64_grids.png}&
                \includegraphics[width=\s]{images/night/LINEAR/bid_j100_64_grids.png}&
                \includegraphics[width=\s]{images/night/PPG/bid_j100_64_grids.png}&
                \includegraphics[width=\s]{images/night/VNG/bid_j100_64_grids.png}\\
                \cmidrule{1-2}
                \multirow{2}{*}[.5em]{{\rotatebox[origin=c]{90}{JPEG 98}}}&
                \raisebox{5pt}{\rotatebox{90}{\tiny Original}} & 
                \includegraphics[width=\s]{images/night/AAHD/iso_j98_64_grids.png}&
                \includegraphics[width=\s]{images/night/AHD/iso_j98_64_grids.png}&
                \includegraphics[width=\s]{images/night/DCB/iso_j98_64_grids.png}&
                \includegraphics[width=\s]{images/night/DHT/iso_j98_64_grids.png}&
                \includegraphics[width=\s]{images/night/LINEAR/iso_j98_64_grids.png}&
                \includegraphics[width=\s]{images/night/PPG/iso_j98_64_grids.png}&
                \includegraphics[width=\s]{images/night/VNG/iso_j98_64_grids.png}\\
                &\rotatebox{90}{\tiny Bidirectional}&
                \includegraphics[width=\s]{images/night/AAHD/bid_j98_64_grids.png}&
                \includegraphics[width=\s]{images/night/AHD/bid_j98_64_grids.png}&
                \includegraphics[width=\s]{images/night/DCB/bid_j98_64_grids.png}&
                \includegraphics[width=\s]{images/night/DHT/bid_j98_64_grids.png}&
                \includegraphics[width=\s]{images/night/LINEAR/bid_j98_64_grids.png}&
                \includegraphics[width=\s]{images/night/PPG/bid_j98_64_grids.png}&
                \includegraphics[width=\s]{images/night/VNG/bid_j98_64_grids.png}\\
                \cmidrule{1-2}
                \multirow{2}{*}[.5em]{{\rotatebox[origin=c]{90}{JPEG 95}}}&
                \raisebox{5pt}{\rotatebox{90}{\tiny Original}} & 
                \includegraphics[width=\s]{images/night/AAHD/iso_j95_64_grids.png}&
                \includegraphics[width=\s]{images/night/AHD/iso_j95_64_grids.png}&
                \includegraphics[width=\s]{images/night/DCB/iso_j95_64_grids.png}&
                \includegraphics[width=\s]{images/night/DHT/iso_j95_64_grids.png}&
                \includegraphics[width=\s]{images/night/LINEAR/iso_j95_64_grids.png}&
                \includegraphics[width=\s]{images/night/PPG/iso_j95_64_grids.png}&
                \includegraphics[width=\s]{images/night/VNG/iso_j95_64_grids.png}\\
                &\rotatebox{90}{\tiny Bidirectional}&
                \includegraphics[width=\s]{images/night/AAHD/bid_j95_64_grids.png}&
                \includegraphics[width=\s]{images/night/AHD/bid_j95_64_grids.png}&
                \includegraphics[width=\s]{images/night/DCB/bid_j95_64_grids.png}&
                \includegraphics[width=\s]{images/night/DHT/bid_j95_64_grids.png}&
                \includegraphics[width=\s]{images/night/LINEAR/bid_j95_64_grids.png}&
                \includegraphics[width=\s]{images/night/PPG/bid_j95_64_grids.png}&
                \includegraphics[width=\s]{images/night/VNG/bid_j95_64_grids.png}\\
                \bottomrule
        \end{tabular}
        \caption{Image r0ea0825ft, in \textsc{grbg} pattern}
\end{subfigure}
\caption{Detection of the method on $64\times64$ blocks on two images, uncompressed and submitted to JPEG compression of quality 100, 98 and 95. At JPEG quality 100 (the highest possible), although the correct pattern is usually found in most blocks of the image, errors between the two dual patterns start to appear. At JPEG quality 98, the method remains globally able to detect the main diagonal, but cannot distinguish the dual patterns anymore. At JPEG quality 95, the algorithm is unable to do any detection, even against bilinear demosaicking. Bidirectional intermediates provide a small boost to JPEG robustness, though it is not enough to make the algorithm reliable to use on JPEG-compressed images.}
\label{fig:jpeg}
\end{figure}


\begin{figure}[ht]
        \centering
        
        \begin{subfigure}[t]{\linewidth}
        \begin{tabular}{ccccccccc}
                \multicolumn{9}{c}{
                        \begin{tabular}{cl}
                                \includegraphics[height=150pt]{images/original/r07cfb432t.jpeg}&
                                \includegraphics[height=150pt]{images/cb.png}
                                \\
                        \end{tabular}
                }\\                                
                && AAHD & AHD & DCB & DHT & Bilinear & PPG & VNG\\
                \midrule
                \multirow{2}{*}[1.8em]{{\rotatebox[origin=c]{90}{Uncompressed}}}&
                \raisebox{5pt}{\rotatebox{90}{\tiny Original}} & 
                \includegraphics[width=\s]{images/flowers/AAHD/iso_64_grids.png}&
                \includegraphics[width=\s]{images/flowers/AHD/iso_64_grids.png}&
                \includegraphics[width=\s]{images/flowers/DCB/iso_64_grids.png}&
                \includegraphics[width=\s]{images/flowers/DHT/iso_64_grids.png}&
                \includegraphics[width=\s]{images/flowers/LINEAR/iso_64_grids.png}&
                \includegraphics[width=\s]{images/flowers/PPG/iso_64_grids.png}&
                \includegraphics[width=\s]{images/flowers/VNG/iso_64_grids.png}\\
                &\rotatebox{90}{\tiny Bidirectional}&
                \includegraphics[width=\s]{images/flowers/AAHD/bid_64_grids.png}&
                \includegraphics[width=\s]{images/flowers/AHD/bid_64_grids.png}&
                \includegraphics[width=\s]{images/flowers/DCB/bid_64_grids.png}&
                \includegraphics[width=\s]{images/flowers/DHT/bid_64_grids.png}&
                \includegraphics[width=\s]{images/flowers/LINEAR/bid_64_grids.png}&
                \includegraphics[width=\s]{images/flowers/PPG/bid_64_grids.png}&
                \includegraphics[width=\s]{images/flowers/VNG/bid_64_grids.png}\\
                \cmidrule{1-2}
                \multirow{2}{*}[1.5em]{{\rotatebox[origin=c]{90}{Noisy $\sigma=5$}}}&
                \raisebox{5pt}{\rotatebox{90}{\tiny Original}} & 
                \includegraphics[width=\s]{images/flowers/AAHD/iso_n5_64_grids.png}&
                \includegraphics[width=\s]{images/flowers/AHD/iso_n5_64_grids.png}&
                \includegraphics[width=\s]{images/flowers/DCB/iso_n5_64_grids.png}&
                \includegraphics[width=\s]{images/flowers/DHT/iso_n5_64_grids.png}&
                \includegraphics[width=\s]{images/flowers/LINEAR/iso_n5_64_grids.png}&
                \includegraphics[width=\s]{images/flowers/PPG/iso_n5_64_grids.png}&
                \includegraphics[width=\s]{images/flowers/VNG/iso_n5_64_grids.png}\\
                &\rotatebox{90}{\tiny Bidirectional}&
                \includegraphics[width=\s]{images/flowers/AAHD/bid_n5_64_grids.png}&
                \includegraphics[width=\s]{images/flowers/AHD/bid_n5_64_grids.png}&
                \includegraphics[width=\s]{images/flowers/DCB/bid_n5_64_grids.png}&
                \includegraphics[width=\s]{images/flowers/DHT/bid_n5_64_grids.png}&
                \includegraphics[width=\s]{images/flowers/LINEAR/bid_n5_64_grids.png}&
                \includegraphics[width=\s]{images/flowers/PPG/bid_n5_64_grids.png}&
                \includegraphics[width=\s]{images/flowers/VNG/bid_n5_64_grids.png}\\
                \cmidrule{1-2}
                \multirow{2}{*}[1.5em]{{\rotatebox[origin=c]{90}{Noisy $\sigma=10$}}}&
                \raisebox{5pt}{\rotatebox{90}{\tiny Original}} & 
                \includegraphics[width=\s]{images/flowers/AAHD/iso_n10_64_grids.png}&
                \includegraphics[width=\s]{images/flowers/AHD/iso_n10_64_grids.png}&
                \includegraphics[width=\s]{images/flowers/DCB/iso_n10_64_grids.png}&
                \includegraphics[width=\s]{images/flowers/DHT/iso_n10_64_grids.png}&
                \includegraphics[width=\s]{images/flowers/LINEAR/iso_n10_64_grids.png}&
                \includegraphics[width=\s]{images/flowers/PPG/iso_n10_64_grids.png}&
                \includegraphics[width=\s]{images/flowers/VNG/iso_n10_64_grids.png}\\
                &\rotatebox{90}{\tiny Bidirectional}&
                \includegraphics[width=\s]{images/flowers/AAHD/bid_n10_64_grids.png}&
                \includegraphics[width=\s]{images/flowers/AHD/bid_n10_64_grids.png}&
                \includegraphics[width=\s]{images/flowers/DCB/bid_n10_64_grids.png}&
                \includegraphics[width=\s]{images/flowers/DHT/bid_n10_64_grids.png}&
                \includegraphics[width=\s]{images/flowers/LINEAR/bid_n10_64_grids.png}&
                \includegraphics[width=\s]{images/flowers/PPG/bid_n10_64_grids.png}&
                \includegraphics[width=\s]{images/flowers/VNG/bid_n10_64_grids.png}\\
                \bottomrule
        \end{tabular}
        \caption{Image r07cfb432t, in \textsc{rggb} pattern. Noise standard deviation from 0(noiseless) to 10, window size $64\times64$.}
        \label{fig:noise:1}
\end{subfigure}
\end{figure}

\begin{figure}[ht]
        \centering
        \ContinuedFloat
        
        \begin{subfigure}[t]{\linewidth}
        \begin{tabular}{ccccccccc}
                \multicolumn{9}{c}{
                        \begin{tabular}{cl}
                                \includegraphics[height=150pt]{images/original/r1c9fdcf4t.jpeg}&
                                \includegraphics[height=150pt]{images/cb.png}
                                \\
                        \end{tabular}
                }\\                                
                && AAHD & AHD & DCB & DHT & Bilinear & PPG & VNG\\
                \midrule
                \multirow{2}{*}[1.8em]{{\rotatebox[origin=c]{90}{Uncompressed}}}&
                \raisebox{5pt}{\rotatebox{90}{\tiny Original}} & 
                \includegraphics[width=\s]{images/tower/AAHD/iso_64_grids.png}&
                \includegraphics[width=\s]{images/tower/AHD/iso_64_grids.png}&
                \includegraphics[width=\s]{images/tower/DCB/iso_64_grids.png}&
                \includegraphics[width=\s]{images/tower/DHT/iso_64_grids.png}&
                \includegraphics[width=\s]{images/tower/LINEAR/iso_64_grids.png}&
                \includegraphics[width=\s]{images/tower/PPG/iso_64_grids.png}&
                \includegraphics[width=\s]{images/tower/VNG/iso_64_grids.png}\\
                &\rotatebox{90}{\tiny Bidirectional}&
                \includegraphics[width=\s]{images/tower/AAHD/bid_64_grids.png}&
                \includegraphics[width=\s]{images/tower/AHD/bid_64_grids.png}&
                \includegraphics[width=\s]{images/tower/DCB/bid_64_grids.png}&
                \includegraphics[width=\s]{images/tower/DHT/bid_64_grids.png}&
                \includegraphics[width=\s]{images/tower/LINEAR/bid_64_grids.png}&
                \includegraphics[width=\s]{images/tower/PPG/bid_64_grids.png}&
                \includegraphics[width=\s]{images/tower/VNG/bid_64_grids.png}\\
                \cmidrule{1-2}
                \multirow{2}{*}[1.5em]{{\rotatebox[origin=c]{90}{Noisy $\sigma=5$}}}&
                \raisebox{5pt}{\rotatebox{90}{\tiny Original}} & 
                \includegraphics[width=\s]{images/tower/AAHD/iso_n5_64_grids.png}&
                \includegraphics[width=\s]{images/tower/AHD/iso_n5_64_grids.png}&
                \includegraphics[width=\s]{images/tower/DCB/iso_n5_64_grids.png}&
                \includegraphics[width=\s]{images/tower/DHT/iso_n5_64_grids.png}&
                \includegraphics[width=\s]{images/tower/LINEAR/iso_n5_64_grids.png}&
                \includegraphics[width=\s]{images/tower/PPG/iso_n5_64_grids.png}&
                \includegraphics[width=\s]{images/tower/VNG/iso_n5_64_grids.png}\\
                &\rotatebox{90}{\tiny Bidirectional}&
                \includegraphics[width=\s]{images/tower/AAHD/bid_n5_64_grids.png}&
                \includegraphics[width=\s]{images/tower/AHD/bid_n5_64_grids.png}&
                \includegraphics[width=\s]{images/tower/DCB/bid_n5_64_grids.png}&
                \includegraphics[width=\s]{images/tower/DHT/bid_n5_64_grids.png}&
                \includegraphics[width=\s]{images/tower/LINEAR/bid_n5_64_grids.png}&
                \includegraphics[width=\s]{images/tower/PPG/bid_n5_64_grids.png}&
                \includegraphics[width=\s]{images/tower/VNG/bid_n5_64_grids.png}\\
                \cmidrule{1-2}
                \multirow{2}{*}[1.5em]{{\rotatebox[origin=c]{90}{Noisy $\sigma=10$}}}&
                \raisebox{5pt}{\rotatebox{90}{\tiny Original}} & 
                \includegraphics[width=\s]{images/tower/AAHD/iso_n10_64_grids.png}&
                \includegraphics[width=\s]{images/tower/AHD/iso_n10_64_grids.png}&
                \includegraphics[width=\s]{images/tower/DCB/iso_n10_64_grids.png}&
                \includegraphics[width=\s]{images/tower/DHT/iso_n10_64_grids.png}&
                \includegraphics[width=\s]{images/tower/LINEAR/iso_n10_64_grids.png}&
                \includegraphics[width=\s]{images/tower/PPG/iso_n10_64_grids.png}&
                \includegraphics[width=\s]{images/tower/VNG/iso_n10_64_grids.png}\\
                &\rotatebox{90}{\tiny Bidirectional}&
                \includegraphics[width=\s]{images/tower/AAHD/bid_n10_64_grids.png}&
                \includegraphics[width=\s]{images/tower/AHD/bid_n10_64_grids.png}&
                \includegraphics[width=\s]{images/tower/DCB/bid_n10_64_grids.png}&
                \includegraphics[width=\s]{images/tower/DHT/bid_n10_64_grids.png}&
                \includegraphics[width=\s]{images/tower/LINEAR/bid_n10_64_grids.png}&
                \includegraphics[width=\s]{images/tower/PPG/bid_n10_64_grids.png}&
                \includegraphics[width=\s]{images/tower/VNG/bid_n10_64_grids.png}\\
                \bottomrule
        \end{tabular}
        \caption{Image r1c9fdcf4t, in \textsc{rggb} pattern. Noise standard deviation from 0(noiseless) to 10, window size $64\times64$.}
        \label{fig:noise:2}
\end{subfigure}
\end{figure}

\begin{figure}[ht]
        \centering
        \ContinuedFloat
        
        \begin{subfigure}[t]{\linewidth}
        \begin{tabular}{ccccccccc}
                \multicolumn{9}{c}{
                        \begin{tabular}{cl}
                                \includegraphics[height=150pt]{images/original/r1c9fdcf4t.jpeg}&
                                \includegraphics[height=150pt]{images/cb.png}
                                \\
                        \end{tabular}
                }\\                                
                && AAHD & AHD & DCB & DHT & Bilinear & PPG & VNG\\
                \midrule
                \multirow{2}{*}[1.5em]{{\rotatebox[origin=c]{90}{\footnotesize $\sigma=5$, $W=64$}}}&
                \raisebox{5pt}{\rotatebox{90}{\tiny Original}} & 
                \includegraphics[width=\s]{images/tower/AAHD/iso_n5_64_grids.png}&
                \includegraphics[width=\s]{images/tower/AHD/iso_n5_64_grids.png}&
                \includegraphics[width=\s]{images/tower/DCB/iso_n5_64_grids.png}&
                \includegraphics[width=\s]{images/tower/DHT/iso_n5_64_grids.png}&
                \includegraphics[width=\s]{images/tower/LINEAR/iso_n5_64_grids.png}&
                \includegraphics[width=\s]{images/tower/PPG/iso_n5_64_grids.png}&
                \includegraphics[width=\s]{images/tower/VNG/iso_n5_64_grids.png}\\
                &\rotatebox{90}{\tiny Bidirectional}&
                \includegraphics[width=\s]{images/tower/AAHD/bid_n5_64_grids.png}&
                \includegraphics[width=\s]{images/tower/AHD/bid_n5_64_grids.png}&
                \includegraphics[width=\s]{images/tower/DCB/bid_n5_64_grids.png}&
                \includegraphics[width=\s]{images/tower/DHT/bid_n5_64_grids.png}&
                \includegraphics[width=\s]{images/tower/LINEAR/bid_n5_64_grids.png}&
                \includegraphics[width=\s]{images/tower/PPG/bid_n5_64_grids.png}&
                \includegraphics[width=\s]{images/tower/VNG/bid_n5_64_grids.png}\\
                \cmidrule{1-2}
                \multirow{2}{*}[1em]{{\rotatebox[origin=c]{90}{\footnotesize $\sigma=5$, $W=128$}}}&
                \raisebox{5pt}{\rotatebox{90}{\tiny Original}} & 
                \includegraphics[width=\s]{images/tower/AAHD/iso_n5_128_grids.png}&
                \includegraphics[width=\s]{images/tower/AHD/iso_n5_128_grids.png}&
                \includegraphics[width=\s]{images/tower/DCB/iso_n5_128_grids.png}&
                \includegraphics[width=\s]{images/tower/DHT/iso_n5_128_grids.png}&
                \includegraphics[width=\s]{images/tower/LINEAR/iso_n5_128_grids.png}&
                \includegraphics[width=\s]{images/tower/PPG/iso_n5_128_grids.png}&
                \includegraphics[width=\s]{images/tower/VNG/iso_n5_128_grids.png}\\
                &\rotatebox{90}{\tiny Bidirectional}&
                \includegraphics[width=\s]{images/tower/AAHD/bid_n5_128_grids.png}&
                \includegraphics[width=\s]{images/tower/AHD/bid_n5_128_grids.png}&
                \includegraphics[width=\s]{images/tower/DCB/bid_n5_128_grids.png}&
                \includegraphics[width=\s]{images/tower/DHT/bid_n5_128_grids.png}&
                \includegraphics[width=\s]{images/tower/LINEAR/bid_n5_128_grids.png}&
                \includegraphics[width=\s]{images/tower/PPG/bid_n5_128_grids.png}&
                \includegraphics[width=\s]{images/tower/VNG/bid_n5_128_grids.png}\\
                \cmidrule{1-2}
                \multirow{2}{*}[1em]{{\rotatebox[origin=c]{90}{\footnotesize $\sigma=5$, $W=256$}}}&
                \raisebox{5pt}{\rotatebox{90}{\tiny Original}} & 
                \includegraphics[width=\s]{images/tower/AAHD/iso_n5_256_grids.png}&
                \includegraphics[width=\s]{images/tower/AHD/iso_n5_256_grids.png}&
                \includegraphics[width=\s]{images/tower/DCB/iso_n5_256_grids.png}&
                \includegraphics[width=\s]{images/tower/DHT/iso_n5_256_grids.png}&
                \includegraphics[width=\s]{images/tower/LINEAR/iso_n5_256_grids.png}&
                \includegraphics[width=\s]{images/tower/PPG/iso_n5_256_grids.png}&
                \includegraphics[width=\s]{images/tower/VNG/iso_n5_256_grids.png}\\
                &\rotatebox{90}{\tiny Bidirectional}&
                \includegraphics[width=\s]{images/tower/AAHD/bid_n5_256_grids.png}&
                \includegraphics[width=\s]{images/tower/AHD/bid_n5_256_grids.png}&
                \includegraphics[width=\s]{images/tower/DCB/bid_n5_256_grids.png}&
                \includegraphics[width=\s]{images/tower/DHT/bid_n5_256_grids.png}&
                \includegraphics[width=\s]{images/tower/LINEAR/bid_n5_256_grids.png}&
                \includegraphics[width=\s]{images/tower/PPG/bid_n5_256_grids.png}&
                \includegraphics[width=\s]{images/tower/VNG/bid_n5_256_grids.png}\\
                \bottomrule
        \end{tabular}
        \caption{Image r1c9fdcf4t, in \textsc{rggb} pattern. Noise standard deviation 5, window sizes from $64\times64$ to $256\times256$}
        \label{fig:noise:3}
\end{subfigure}
\caption{Robustness of the method to additive white Gaussian noise (AWGN). Figures~\ref{fig:noise:1} and \ref{fig:noise:2} show results with $64\times64$ windows, on noiseless images and with AWGN of standard deviation 5 and 10. Figure~\ref{fig:noise:3} shows results with AWGN of standard deviation 5, with window sizes $64\times64$, $128\times128$ and $256\times256$. Because the noise is independant to the image, it does not create locally coherent errors that can hardly be distinguished from forgeries. However, the probabilities of a sampled or interpolated pixel being an intermediate value go closer to one another as more noise is added, making the detection harder. As seen in Figure~\ref{fig:noise:3}, using bigger windows can alleviate this difficulty by providing more samples (at the cost of potentially missing smaller forgeries).}
\label{fig:noise}
\end{figure}

\begin{figure}[ht]
        \centering
        
        \begin{subfigure}[t]{\linewidth}
        \begin{tabular}{ccccccccc}
                \multicolumn{9}{c}{
                        \begin{tabular}{cl}
                                \includegraphics[height=90pt]{images/original/r0e04cc91t.jpeg}&
                                \includegraphics[height=90pt]{images/cb.png}
                                \\
                        \end{tabular}
                }\\                                
                && AAHD & AHD & DCB & DHT & Bilinear & PPG & VNG\\
                \midrule
                \multirow{2}{*}[1.2em]{{\rotatebox[origin=c]{90}{Base image}}}&
                \raisebox{5pt}{\rotatebox{90}{\tiny Original}} & 
                \includegraphics[width=\s]{images/lake/AAHD/iso_64_grids.png}&
                \includegraphics[width=\s]{images/lake/AHD/iso_64_grids.png}&
                \includegraphics[width=\s]{images/lake/DCB/iso_64_grids.png}&
                \includegraphics[width=\s]{images/lake/DHT/iso_64_grids.png}&
                \includegraphics[width=\s]{images/lake/LINEAR/iso_64_grids.png}&
                \includegraphics[width=\s]{images/lake/PPG/iso_64_grids.png}&
                \includegraphics[width=\s]{images/lake/VNG/iso_64_grids.png}\\
                &\rotatebox{90}{\tiny Bidirectional}&
                \includegraphics[width=\s]{images/lake/AAHD/bid_64_grids.png}&
                \includegraphics[width=\s]{images/lake/AHD/bid_64_grids.png}&
                \includegraphics[width=\s]{images/lake/DCB/bid_64_grids.png}&
                \includegraphics[width=\s]{images/lake/DHT/bid_64_grids.png}&
                \includegraphics[width=\s]{images/lake/LINEAR/bid_64_grids.png}&
                \includegraphics[width=\s]{images/lake/PPG/bid_64_grids.png}&
                \includegraphics[width=\s]{images/lake/VNG/bid_64_grids.png}\\
                \cmidrule{1-2}
                \multirow{2}{*}[1.5em]{{\rotatebox[origin=c]{90}{Median filter}}}&
                \raisebox{5pt}{\rotatebox{90}{\tiny Original}} & 
                \includegraphics[width=\s]{images/lake/AAHD/iso_med_64_grids.png}&
                \includegraphics[width=\s]{images/lake/AHD/iso_med_64_grids.png}&
                \includegraphics[width=\s]{images/lake/DCB/iso_med_64_grids.png}&
                \includegraphics[width=\s]{images/lake/DHT/iso_med_64_grids.png}&
                \includegraphics[width=\s]{images/lake/LINEAR/iso_med_64_grids.png}&
                \includegraphics[width=\s]{images/lake/PPG/iso_med_64_grids.png}&
                \includegraphics[width=\s]{images/lake/VNG/iso_med_64_grids.png}\\
                &\rotatebox{90}{\tiny Bidirectional}&
                \includegraphics[width=\s]{images/lake/AAHD/bid_med_64_grids.png}&
                \includegraphics[width=\s]{images/lake/AHD/bid_med_64_grids.png}&
                \includegraphics[width=\s]{images/lake/DCB/bid_med_64_grids.png}&
                \includegraphics[width=\s]{images/lake/DHT/bid_med_64_grids.png}&
                \includegraphics[width=\s]{images/lake/LINEAR/bid_med_64_grids.png}&
                \includegraphics[width=\s]{images/lake/PPG/bid_med_64_grids.png}&
                \includegraphics[width=\s]{images/lake/VNG/bid_med_64_grids.png}\\
                \bottomrule
        \end{tabular}
                \caption{Image r0e04cc91t, in \textsc{rggb} pattern.}
\end{subfigure}

        \begin{subfigure}[t]{\linewidth}
        \begin{tabular}{ccccccccc}
                \multicolumn{9}{c}{
                        \begin{tabular}{cl}
                                \includegraphics[height=90pt]{images/original/r1a0f5585t.jpeg}&
                                \includegraphics[height=90pt]{images/cb.png}
                                \\
                        \end{tabular}
                }\\                                
                && AAHD & AHD & DCB & DHT & Bilinear & PPG & VNG\\
                \midrule
                \multirow{2}{*}[1.2em]{{\rotatebox[origin=c]{90}{Base image}}}&
                \raisebox{5pt}{\rotatebox{90}{\tiny Original}} & 
                \includegraphics[width=\s]{images/windmill/AAHD/iso_64_grids.png}&
                \includegraphics[width=\s]{images/windmill/AHD/iso_64_grids.png}&
                \includegraphics[width=\s]{images/windmill/DCB/iso_64_grids.png}&
                \includegraphics[width=\s]{images/windmill/DHT/iso_64_grids.png}&
                \includegraphics[width=\s]{images/windmill/LINEAR/iso_64_grids.png}&
                \includegraphics[width=\s]{images/windmill/PPG/iso_64_grids.png}&
                \includegraphics[width=\s]{images/windmill/VNG/iso_64_grids.png}\\
                &\rotatebox{90}{\tiny Bidirectional}&
                \includegraphics[width=\s]{images/windmill/AAHD/bid_64_grids.png}&
                \includegraphics[width=\s]{images/windmill/AHD/bid_64_grids.png}&
                \includegraphics[width=\s]{images/windmill/DCB/bid_64_grids.png}&
                \includegraphics[width=\s]{images/windmill/DHT/bid_64_grids.png}&
                \includegraphics[width=\s]{images/windmill/LINEAR/bid_64_grids.png}&
                \includegraphics[width=\s]{images/windmill/PPG/bid_64_grids.png}&
                \includegraphics[width=\s]{images/windmill/VNG/bid_64_grids.png}\\
                \cmidrule{1-2}
                \multirow{2}{*}[1.5em]{{\rotatebox[origin=c]{90}{Median filter}}}&
                \raisebox{5pt}{\rotatebox{90}{\tiny Original}} & 
                \includegraphics[width=\s]{images/windmill/AAHD/iso_med_64_grids.png}&
                \includegraphics[width=\s]{images/windmill/AHD/iso_med_64_grids.png}&
                \includegraphics[width=\s]{images/windmill/DCB/iso_med_64_grids.png}&
                \includegraphics[width=\s]{images/windmill/DHT/iso_med_64_grids.png}&
                \includegraphics[width=\s]{images/windmill/LINEAR/iso_med_64_grids.png}&
                \includegraphics[width=\s]{images/windmill/PPG/iso_med_64_grids.png}&
                \includegraphics[width=\s]{images/windmill/VNG/iso_med_64_grids.png}\\
                &\rotatebox{90}{\tiny Bidirectional}&
                \includegraphics[width=\s]{images/windmill/AAHD/bid_med_64_grids.png}&
                \includegraphics[width=\s]{images/windmill/AHD/bid_med_64_grids.png}&
                \includegraphics[width=\s]{images/windmill/DCB/bid_med_64_grids.png}&
                \includegraphics[width=\s]{images/windmill/DHT/bid_med_64_grids.png}&
                \includegraphics[width=\s]{images/windmill/LINEAR/bid_med_64_grids.png}&
                \includegraphics[width=\s]{images/windmill/PPG/bid_med_64_grids.png}&
                \includegraphics[width=\s]{images/windmill/VNG/bid_med_64_grids.png}\\
                \bottomrule
        \end{tabular}
        \caption{Image r1a0f5585t, in \textsc{rggb} pattern}
\end{subfigure}
\caption{Results of the method on $64\times64$ blocks on two images, unprocessed and median-filtered. The median filter shifts the intermediate values on the green channel, thus confusing the algorithm on the diagonal pattern. Consequently, with the AAHD and DHT algorithms, which already shift the green channel intermediate values into the sampled pixels, the algorithms makes better detection after median-filtering than on the unprocessed image.}
\label{fig:median}
\end{figure}

\clearpage
\subsection{Image forgery detection}
The ultimate goal of the method is to find mosaic inconsistencies in an image. We use forgeries from the Trace database~\cite{trace} to evaluate the method. For the quantitative experiments, we use the CFA grid with exomasks dataset. For the qualitative experiments, we use samples from both the CFA grid and CFA algorithm datasets.\qb{TBD: describe the database.}

Except where specified otherwise, quantitative experiments are done with the Matthews Correlation Coefficient (MCC). For some results, we also provide the Intersection over Union (IoU), the F1 score and the Precision and Recall. All metrics are computed independantly on each image, then averaged across all images. Results tables with quantitative experiments can be found in Table~\ref{tab:quantitative}\qb{explain metrics}
\begin{table}[ht]
    \centering
        \begin{subfigure}[b]{.48\linewidth}
                \centering
            \begin{tabular}{lccc}
                    \toprule
                    &MCC&IoU&F1\\
                    \midrule
                    \scriptsize Isotropic, no thresholding&0.518&0.490&0.573\\
                    \scriptsize Isotropic, hysteresis&0.598&0.556&0.633\\
                    \scriptsize Bidirectional, no thresholding&0.543&0.515&0.595\\
                    \scriptsize Bidirectional, hysteresis&0.588&0.550&0.628\\
                    \cmidrule{1-1}
                    Bammey~\cite{bammey20}&0.682&0.617&0.702\\
                   \bottomrule
            \end{tabular}
                \caption{Results with isotropic and bidirectional intermediate values, with and without hysteresis thresholding, compared with Bammey~\cite{bammey20}. Both the presented method and Bammey are used on $32\times32$ windows.}
        \end{subfigure}\hfill%
        \begin{subfigure}[b]{.48\linewidth}
                \centering
                \begin{tabular}{lccc}
                        \toprule
                        &\scriptsize All images &\scriptsize  Same diagonal &\scriptsize  Different diagonal\\
                        \midrule
                        \scriptsize Main grid & 0.574 & 0.509 & 0.554\\
                        \scriptsize Diagonal & 0.426 & -0.001 & 0.671\\
                        \scriptsize Combined & 0.570 & 0.503 & 0.637\\
                        \bottomrule
                \end{tabular}
                \caption{Influence of using only main grid inconsistencies, diagonal inconsistencies and their combination (pointwise maximum of the two detection maps), on the full database, and when only looking at images whose authentic and forged parts share/do not share the same diagonal. The diagonal is shared in 364 out of the 1000 images of the dataset.}
        \end{subfigure}

        \vspace{1em}

        \begin{subfigure}[b]{\linewidth}
                \centering
                \begin{tabular}{lcccccccc}
                        \toprule
                        \scriptsize Algorithm&All& AAHD & AHD & DCB & DHT & Bilinear & PPG & VNG\\
                        \scriptsize\#Images&1000&126&138&133&155&154&147&147\\
                        \midrule
                        \scriptsize Score&0.589&0.311&0.705&0.754&0.338&0.732&0.709&0.561\\
                        \bottomrule
                \end{tabular}
                \caption{Results of the presented method depending on how the image was demosaicked. The method is used with bidirectional filters, on $64\times64$ windows, with hysteresis thresholding and combining the main grid and diagonal inconsistencies. Even though the method finds the wrong diagonal with the AAHD and DHT algorithms, it is consistent in doing so, and can thus still detect some forgeries, though not as well as against other demosaicking algorithms.}
        \end{subfigure}

        \begin{subfigure}[b]{.56\linewidth}
                \centering
                \begin{tabular}{lccccc}
                        \toprule
                        &MCC&IoU&F1&\scriptsize Precision&\scriptsize Recall\\
                        \midrule
                        \scriptsize No thresholding & 0.522 & 0.490 & 0.578 & 0.579 & 0.706\\
                        \scriptsize Hysteresis & 0.589 & 0.573 & 0.620 & 0.663 & 0.639\\
                        \bottomrule
                \end{tabular}
                \caption{Results with different metrics, with and without hysteresis thresholding. The method is used with bidirectional filters, on $64\times64$ windows and combining the main and diagonal inconsistencies. Even though thresholding slightly lowers the recall, its gain is precision is much larger, thus yielding better MCC, IoU and F1 scores.}
        \end{subfigure}\hfill%
        \begin{subfigure}[b]{.4\linewidth}
                \centering
                \begin{tabular}{lc}
                        \toprule
                        \scriptsize Window size & MCC\\
                        \midrule
                        16 & 0.528\\
                        32 & 0.588\\
                        64 & 0.589\\
                        128 & 0.501\\
                        \bottomrule
                \end{tabular}
                \caption{Results with different window sizes. The method is used with bidirectional filters, hysteresis thresholding and combination of the main and diagonal inconsistencies.}
        \end{subfigure}
        \caption{Quantitative experiments on the Trace database~\cite{trace}}
        \label{tab:quantitative}
\end{table}


\begin{figure}[ht]
        \begin{subfigure}[t]{.235\linewidth}
                \includegraphics[width=\linewidth]{images/aahd_dcb/image.jpeg}
                \caption{Input image}
        \end{subfigure}\hfill%
        \begin{subfigure}[t]{.235\linewidth}
                \includegraphics[width=\linewidth]{images/aahd_dcb/mask_a.png}
                \caption{Ground Truth}
        \end{subfigure}\hfill%
        \begin{subfigure}[t]{.235\linewidth}
                \includegraphics[width=\linewidth]{images/aahd_dcb/out_inconsistent_diag_thr_fig.png}
                \caption{Detected inconsistencies on the diagonal.}
        \end{subfigure}\hfill%
        \begin{subfigure}[t]{.275\linewidth}
                \includegraphics[width=\linewidth]{images/aahd_dcb/out_diff_diag.png}
                \caption{Normalized difference between the two diagonal patterns.}
        \end{subfigure}
        \caption{Image r040b3002t of the \texttt{cfa\_alg} dataset, with exomask. The authentic region is demosaiced with the \textsc{aahd} algorithm in the GRBG pattern, the forged region is demosaiced with the DCB algorithm in the \textsc{bggr} pattern. Because the method consistently finds the wrong diagonal on AAHD-demosaiced images, but detects the correct diagonal on DCB-demosaiced images, it believes that the two regions share the same diagonal, even though they do not.}
        \label{fig:aahd_dcb}
\end{figure}

\begin{figure}[ht]
        \begin{subfigure}[t]{.3015\linewidth}
                \includegraphics[width=\linewidth]{images/aahd_nodual/image.jpeg}
                \caption{Input image}
        \end{subfigure}\hfill%
        \begin{subfigure}[t]{.3015\linewidth}
                \includegraphics[width=\linewidth]{images/aahd_nodual/out_inconsistent_diag_thr_fig.png}
                \caption{Detected inconsistencies on the diagonal.}
        \end{subfigure}\hfill%
        \begin{subfigure}[t]{.3585\linewidth}
                \includegraphics[width=\linewidth]{images/aahd_nodual/out_diff_diag.png}
                \caption{Normalized difference between the two diagonal patterns.}
        \end{subfigure}

        \begin{subfigure}[t]{.301\linewidth}
                \includegraphics[width=\linewidth]{images/aahd_nodual/mask_s.png}
                \caption{Ground truth}
        \end{subfigure}\hfill%
        \begin{subfigure}[t]{.301\linewidth}
                \includegraphics[width=\linewidth]{images/aahd_nodual/out_inconsistent_grid_thr_fig.png}
                \caption{Detected inconsistencies on the full pattern.}
        \end{subfigure}\hfill%
        \begin{subfigure}[t]{.365\linewidth}
                \includegraphics[width=\linewidth]{images/aahd_nodual/out_diff_grid.png}
                \caption{Normalized difference between the two patterns sharing the same diagonal.}
        \end{subfigure}
        \caption{Image r15919202t of the \texttt{cfa\_grid} dataset, with endomask. Both regions are demosaiced with the AAHD algorithm, the authentic region in the \textsc{grbg} pattern, the forged region in the \textsc{bggr} pattern. Although the method finds the wrong diagonal in both regions, it still finds that the two regions use a different diagonal. However, because the diagonal is wrong, the rest of the pattern cannot be accurately detected.}
\end{figure}

\begin{figure}[ht]
        \begin{subfigure}[t]{.22\linewidth}
                \includegraphics[width=\linewidth]{images/forged_house/image.jpeg}
                \caption{Input image}
        \end{subfigure}\hfill%
        \begin{subfigure}[t]{.258\linewidth}
                \includegraphics[width=\linewidth]{images/forged_house/out_diff_diag.png}
                \caption{Normalized difference between the two diagonal patterns.}
        \end{subfigure}\hfill%
        \begin{subfigure}[t]{.245\linewidth}
                \includegraphics[width=\linewidth]{images/forged_house/out_inconsistent_diag_fig.png}
                \caption{Inconsistencies in the diagonal (not thresholded)}
        \end{subfigure}\hfill%
        \begin{subfigure}[t]{.245\linewidth}
                \includegraphics[width=\linewidth]{images/forged_house/out_inconsistent_diag_thr_fig.png}
                \caption{Inconsistencies on the diagonal (thresholded)}
        \end{subfigure}

        \begin{subfigure}[t]{.22\linewidth}
                \includegraphics[width=\linewidth]{images/forged_house/mask_s.png}
                \caption{Ground Truth}
        \end{subfigure}\hfill%
        \begin{subfigure}[t]{.258\linewidth}
                \includegraphics[width=\linewidth]{images/forged_house/out_diff_grid.png}
                \caption{Normalized difference between the two patterns sharing the detected diagonal.}
        \end{subfigure}\hfill%
        \begin{subfigure}[t]{.245\linewidth}
                \includegraphics[width=\linewidth]{images/forged_house/out_inconsistent_grid_fig.png}
                \caption{Inconsistencies on the full pattern (not thresholded)}
        \end{subfigure}\hfill%
        \begin{subfigure}[t]{.245\linewidth}
                \includegraphics[width=\linewidth]{images/forged_house/out_inconsistent_grid_thr_fig.png}
                \caption{Inconsistencies on the full pattern (thresholded)}
        \end{subfigure}
        \caption{Image r0a966704t of the \texttt{cfa\_alg} dataset, with endomask. Although thresholding gets rid of most detections, the texture on the roof still confuses the algorithm into finding the wrong grid on the detected diagonal.}
\end{figure}





%------------------------------------------------------------------------------
\section{Conclusion}

Here is the conclusion

%------------------------------------------------------------------------------
\section*{Acknowledgment}

\fi
%------------------------------------------------------------------------------
\section*{Image Credits}
{\small\flushleft

}

%------------------------------------------------------------------------------
\bibliographystyle{siam}
\bibliography{article}

\end{document}
%------------------------------------------------------------------------------
